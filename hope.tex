%% Commands for TeXCount
%TC:macro \cite [option:text,text]
%TC:macro \citep [option:text,text]
%TC:macro \citet [option:text,text]
%TC:envir table 0 1
%TC:envir table* 0 1
%TC:envir tabular [ignore] word
%TC:envir displaymath 0 word
%TC:envir math 0 word
%TC:envir comment 0 0

\documentclass[acmsmall,screen,review]{acmart}

\usepackage{syntax}
\renewcommand{\syntleft}{\normalfont\itshape}
\renewcommand{\syntright}{\normalfont\itshape}

\usepackage{prftree}

\usepackage{listings}
\usepackage{lstautogobble}
\usepackage{xcolor} % \usepackage[dvipsnames]{xcolor}
\usepackage{caption}
\usepackage{subcaption}
\usepackage{fancyvrb}
\usepackage{enumitem}
\usepackage{string-diagrams}
\usepackage{cancel}
\usepackage{thmtools}
\usepackage{pifont}
\usepackage[export]{adjustbox}
\usetikzlibrary{calc}

\lstset{ %
  autogobble=true,
}

\definecolor{codegreen}{rgb}{0,0.6,0}
\definecolor{codegray}{rgb}{0.5,0.5,0.5}
\definecolor{codepurple}{rgb}{0.58,0,0.82}
\definecolor{backcolour}{rgb}{0.95,0.95,0.92}

\lstdefinestyle{mystyle}{
%    backgroundcolor=\color{backcolour},   
    commentstyle=\color{codegreen},
    keywordstyle=\color{magenta},
    numberstyle=\tiny\color{codegray},
    stringstyle=\color{codepurple},
    basicstyle=\ttfamily\footnotesize,
    breakatwhitespace=false,         
    breaklines=true,                 
    captionpos=b,                    
    keepspaces=true,                 
    numbers=left,                    
    numbersep=5pt,                  
    showspaces=false,                
    showstringspaces=false,
    showtabs=false,                  
    tabsize=2
}

\lstset{style=mystyle}

\newcounter{todos}
\newcommand{\TODO}[1]{{
  \stepcounter{todos}
  \begin{center}\large{\textcolor{red}{\textbf{TODO \arabic{todos}:} #1}}\end{center}
}}

\newcommand{\todo}[1]{\stepcounter{todos} \textcolor{red}{\textbf{TODO \arabic{todos}}: #1}}

% Math fonts
\newcommand{\mc}[1]{\ensuremath{\mathcal{#1}}}
\newcommand{\mb}[1]{\ensuremath{\mathbf{#1}}}
\newcommand{\mbb}[1]{\ensuremath{\mathbb{#1}}}
\newcommand{\ms}[1]{\ensuremath{\mathsf{#1}}}

\newcommand{\kwms}[1]{\textcolor{violet}{\ms{#1}}}
\newcommand{\lbms}[1]{{\ms{#1}}}

% Math
\newcommand{\nats}{\mathbb{N}}

% Syntax atoms
\newcommand{\lbl}[1]{{`#1}}
\newcommand{\lto}{:}
\newcommand{\linl}[1]{\iota_l\;{#1}}
\newcommand{\linr}[1]{\iota_r\;{#1}}
\newcommand{\labort}[1]{\ms{abort}\;{#1}}

% Syntax
\newcommand{\letexpr}[3]{\ensuremath{\ms{let}\;#1 = #2;\;#3}}
\newcommand{\caseexpr}[5]{\ms{case}\;#1\;\{\linl{#2} \lto #3, \linr{#4} \lto #5\}}
\newcommand{\letstmt}[3]{\ensuremath{\ms{let}\;#1 = #2; #3}}
\newcommand{\brb}[2]{\ms{br}\;#1\;#2}
\newcommand{\ite}[3]{\ms{if}\;#1\;\{#2\}\;\ms{else}\;\{#3\}}
\newcommand{\casestmt}[5]{\ms{case}\;#1\;\{\linl{#2} \lto #3, \linr{#4} \lto #5\}}
\newcommand{\loopstmt}[4]{\ms{loop}\;#1\;\{#2(#3) \lto #4\}}
\newcommand{\awhere}[2]{#1\;\ms{where}_{\ms{nonrec}}\;#2}
\newcommand{\cwhere}[2]{#1\;\ms{where}_{\ms{rec}}\;#2}
\newcommand{\where}[2]{#1\;\ms{where}\;#2}
\newcommand{\wbranch}[3]{#1(#2) \lto \{#3\}}
\newcommand{\cfgsubst}[1]{\ms{cfgs}\;\{#1\}}
\newcommand{\wseq}[2]{{#1} \mathbin{{;}{;}} {#2}}
\newcommand{\rupg}[1]{{#1}^\upharpoonright}
\newcommand{\lupg}[1]{{#1}^\upharpoonleft}
\newcommand{\liter}[3]{\ms{iter}\;#1\;\{ \linr{#2} \lto #3 \}}
\newcommand{\einf}[1]{#1 \in \mc{E}^\infty}

\newcommand{\ncaseexpr}[3]{\ms{case}_{#1}\;#2\;\{#3\}}
\newcommand{\splitexpr}[3]{\ms{split}_{#1; #2}(#3)}
\newcommand{\webranch}[3]{#1(#2) \lto #3}

% Judgements
\newcommand{\qsp}[4]{#1 \vdash #2 = #3 + #4}
\newcommand{\qwk}[4]{#1 \vdash #2 \geq #3 + #4}
\newcommand{\swk}[3]{#1 \mapsto #2 ; #3}
\newcommand{\cwk}[2]{#1 \mapsto #2}
\newcommand{\lwk}[3]{#1 \vdash #2 \rightsquigarrow #3}
\newcommand{\thyp}[3]{#1 : {#2}^{#3}}
\newcommand{\bhyp}[2]{#1 : #2}
\newcommand{\lhyp}[2]{#1(#2)}
\newcommand{\rle}[1]{{\scriptsize\textsf{#1}}}
\newcommand{\qbc}[2]{(#1) , #2}

\newcommand{\hasty}[4]{#1 \vdash_{#2} #3: {#4}}
\newcommand{\haslb}[4]{#1 \vdash_{#2} #3 \rhd #4}

\newcommand{\ahasty}[4]{#1 \vdash_{#2}^{\ms{anf}} #3 : {#4}}
\newcommand{\thaslb}[3]{#1 \vdash^{\ms{t}}_{\ms{ssa}} #2 \rhd #3}
\newcommand{\ahaslb}[3]{#1 \vdash^{\ms{anf}} #2 \rhd #3}
\newcommand{\bhaslb}[3]{#1 \vdash^{\ms{b}}_{\ms{ssa}} #2 \rhd #3}
% \newcommand{\chaslb}[3]{#1 \vdash^{\ms{c}}_{\ms{ssa}} #2 \rhd #3}

\newcommand{\shaslb}[3]{#1 \vdash^{\ms{s}} #2 \rhd #3}

\newcommand{\isop}[4]{#1 \in \mc{I}_{#4}(#2, #3)}
\newcommand{\issubst}[4]{#1 \vdash_{#2} #3 \rhd #4}
\newcommand{\lbsubst}[4]{#1 \vdash #2: #3 \rightsquigarrow #4}
\newcommand{\teqv}{\approx}
\newcommand{\tref}{\twoheadrightarrow}
\newcommand{\antitref}{\twoheadleftarrow}
\newcommand{\tmle}[5]{#1 \vdash_{#2} #3 \tref #4 : {#5}}
\newcommand{\tmlep}[6]{#1 \vdash_{#2} #3 \tref^{#6} #4 : {#5}}
\newcommand{\tmeq}[5]{#1 \vdash_{#2} #3 \teqv #4 : {#5}}
\newcommand{\lbeq}[4]{#1 \vdash #2 \teqv #3 \rhd {#4}}
\newcommand{\tmseq}[4]{\issubst{#1 \teqv #2}{#3}{#4}}
\newcommand{\lbseq}[5]{\lbsubst{#1 \teqv #2}{#3}{#4}{#5}}
\newcommand{\brle}[1]{{\textsf{#1}}}

\newcommand{\tossa}[2]{\ms{SSA}(#1 \Rightarrow #2)}
\newcommand{\ssalet}[3]{\ms{SSA}_{\ms{let}}(#1, #2, #3)}
\newcommand{\toanf}[1]{\ms{ANF}(#1)}
\newcommand{\anflet}[3]{\ms{ANF}_{\ms{let}}(#1, #2, #3)}
\newcommand{\toterm}[1]{\ms{Term}(#1)}
\newcommand{\etoty}[1]{[#1]}
\newcommand{\ctoty}[1]{[#1]}
\newcommand{\ltoty}[2]{[#1]}
\newcommand{\dltoty}[2]{[#1; #2]}

% Denotational semantics
\newcommand{\dnt}[1]{\llbracket{#1}\rrbracket}
\newcommand{\ednt}[1]{\left\llbracket{#1}\right\rrbracket}
\newcommand{\tmor}[1]{{!}_{#1}}
\newcommand{\dmor}[1]{{\Delta}_{#1}}
\newcommand{\entrymor}[3]{\ms{esem}_{#1, #3}(#2)}
\newcommand{\loopmor}[3]{\ms{lsem}_{#1, #3}(#2)}
\newcommand{\substpure}[1]{#1\;\ms{pure}}

% Comonadic lore
\newcommand{\lmor}[1]{\ms{let}(#1)}
\newcommand{\envcom}[2]{{#1}_{#2 \otimes \cdot}}
\newcommand{\rlmor}[1]{\ms{rlet}(#1)}
\newcommand{\rcase}[1]{\ms{rcase}(#1)}
\newcommand{\rfix}[1]{\ms{rfix}(#1)}
\newcommand{\rseq}[3]{#2 \gg_{#1} #3}
\newcommand{\envpil}[1]{\pi_{{#1}l}}
\newcommand{\envpir}[1]{\pi_{{#1}r}}

\newcommand{\toenv}[2]{\ms{env}_{#1}(#2)}
\newcommand{\envcop}[3]{[#2, #3]_{#1}}
\newcommand{\envinr}[1]{\iota^{#1}_{r}}
\newcommand{\envinl}[1]{\iota^{#1}_{l}}
\newcommand{\envtn}[3]{{#2} \otimes_{#1} {#3}}

% Composition
\newcommand{\invar}{\square}
\newcommand{\outlb}{\blacksquare}
\newcommand{\pckd}[1]{\langle #1 \rangle}

% Weak memory
\newcommand{\bufloc}[1]{\overline{#1}}

% Branding
\newcommand{\subiterexp}{\texorpdfstring{\(\lambda_{\ms{iter}}\)}{lambda-iter}}
\newcommand{\isotopessa}{\(\lambda_{\ms{SSA}}\)}
\newcommand{\thsubiter}[1]{\ms{Th}(#1)}

% Formalization pointers
\newcommand{\formalizedas}[1]{Formalized as \texttt{#1}}

% Commutativity
\newcommand{\rightmove}{\rightharpoonup}
\newcommand{\leftmove}{\leftharpoondown}
\newcommand{\slides}{\rightleftharpoons}

% Quantities
\newcommand{\zeroq}{0}
\newcommand{\oneq}{1}
\newcommand{\delq}{1^?}
\newcommand{\cpyq}{\omega^+}
\newcommand{\topq}{\omega}
\newcommand{\zeroqv}[1]{#1^\uparrow}
\newcommand{\dwnqv}[1]{\downarrow(#1)}

% Operators
\newcommand{\pto}{\rightharpoonup}
\newcommand{\alquant}{\ms{q}}
\newcommand{\alcount}{\sharp}
\newcommand{\aldquant}{\bar{\ms{q}}}
\newcommand{\varcount}[2]{\sharp(#1, #2)}

% Truth tables
\newcommand{\cmark}{\textcolor{green}{\ding{51}}}%
\newcommand{\xmark}{\textcolor{red}{\ding{55}}}%
% \newcommand{\cmark}{\textcolor{Green}{\ding{51}}}%
% \newcommand{\xmark}{\textcolor{BrickRed}{\ding{55}}}%


%% Rights management information.  This information is sent to you
%% when you complete the rights form.  These commands have SAMPLE
%% values in them; it is your responsibility as an author to replace
%% the commands and values with those provided to you when you
%% complete the rights form.
\setcopyright{acmcopyright}
\copyrightyear{2025}
\acmYear{2025}
\acmDOI{XXXXXXX.XXXXXXX}

%%
%% These commands are for a JOURNAL article.
% \acmJournal{JACM}
% \acmVolume{37}
% \acmNumber{4}
% \acmArticle{111}
% \acmMonth{8}

%%
%% Submission ID.
%% Use this when submitting an article to a sponsored event. You'll
%% receive a unique submission ID from the organizers
%% of the event, and this ID should be used as the parameter to this command.
%%\acmSubmissionID{123-A56-BU3}

\begin{document}

\title{Sound and Complete Refinement with Substructural Effects}

\author{Jad Ghalayini}
\email{jeg74@cl.cam.ac.uk}
\orcid{0000-0002-6905-1303}

\author{Neel Krishnaswami}
\email{nk480@cl.cam.ac.uk}
\orcid{0000-0003-2838-5865}

\begin{CCSXML}
  <ccs2012>
  <concept>
  <concept_id>10003752.10010124.10010131.10010133</concept_id>
  <concept_desc>Theory of computation~Denotational semantics</concept_desc>
  <concept_significance>500</concept_significance>
  </concept>
  <concept>
  <concept_id>10003752.10010124.10010131.10010137</concept_id>
  <concept_desc>Theory of computation~Categorical semantics</concept_desc>
  <concept_significance>500</concept_significance>
  </concept>
  <concept>
  <concept_id>10003752.10003790.10011740</concept_id>
  <concept_desc>Theory of 
  Just like for branching control-flow, we also require an additional condition to ensure that our
  iteration operator is compatible with our premonoidal structure. Specifically, we would like to be
  able to ``thread'' values through our loop bodies; i.e., the following two programs should be
  equivalent for \emph{pure} $c$:
  $$
  (\liter{a}{x}{b}, c) \approx \liter{(a, c)}{(x, y)}
    {\caseexpr{b}{z}{\linl{(z, y)}}{z}{\linr{(z, y)}}}
  $$
  This corresponds to requiring our Conway iteration operator to be \emph{strong}, defined as follows:
  \begin{definition}[Strong Conway Iteration Operator]
    If $\mc{C}$ is distributcomputation~Type theory</concept_desc>
  <concept_significance>500</concept_significance>
  </concept>
  </ccs2012>
\end{CCSXML}

\ccsdesc[500]{Theory of computation~Denotational semantics}
\ccsdesc[500]{Theory of computation~Categorical semantics}
\ccsdesc[500]{Theory of computation~Type theory}

%%
%% Keywords. The author(s) should pick words that accurately describe
%% the work being presented. Separate the keywords with commas.
\keywords{SSA, Categorical Semantics, Elgot Structure, Effectful Category}

% \received{20 February 2007}
% \received[revised]{12 March 2009}
% \received[accepted]{5 June 2009}

\maketitle

Verification of compiler optimizations typically proceeds along two broad paths, corresponding to
the chosen semantic framework. On one hand, operational semantics allow verification through
simulation proofs: a transformed program is shown to simulate the behavior of the original. A
classic example is \citet{vellvm-12}, which parametrizes LLVM's operational semantics by memory
models and proves transformations individually sound via bespoke simulation arguments. On the other
hand, denotational semantics enables proving optimizations correct by establishing semantic equality
or refinement between original and transformed programs. Monadic semantics have proven effective
even for complex effects; for instance, monadic Brookes-style semantics have been successfully
applied to weak memory models such as release-acquire and TSO, as seen in recent work by
\citet{release-acquire} and \citet{jagadeesan-brookes-relaxed-12}.

However, both these approaches come with limitations. Simulation proofs typically involve repeated
work for each optimization, reducing reusability—especially when new side effects or memory models
arise. Conversely, constructing a comprehensive denotational model for a specific effect system
often demands significant effort, resulting in proofs tightly coupled to the particulars of that
model. Such proofs rarely generalize, requiring deep familiarity with complex model details.

Interestingly, many practical compiler optimizations are expressible more abstractly through simpler
properties of the underlying effects, namely: (1) how different effects commute, (2) peephole
optimizations on effectful operations, and crucially, (3) the linearity characteristics of
effects—whether they can be duplicated or discarded safely. In particular, these properties allow
one to reason about the soundness of \emph{directed substitution}, i.e., whether
$$
\letexpr{x}{a}{b} \tref [a/x]b
$$
or
$$
[a/x]b \tref \letexpr{x}{a}{b}
$$
where $\tref$ indicates refinement.

To illustrate these ideas concretely, consider undefined behavior (UB), as exhibited by division by
zero. We have a straightforward refinement:
\[
(\ms{div}\;y\;z, \ms{div}\;y\;z) \tref \letexpr{x}{\ms{div}\;y\;z}{(x, x)}
\]
This refinement expresses that UB is fusable. Moreover, the reverse refinement also holds, showing
that UB is duplicable. \citet{fuhrmann-direct-1999} classifies effects like UB, which are both
fusable and duplicable, as copyable. By contrast, nondeterminism is fusable but not duplicable, as
the refinement:
\[
(\ms{nondet}, \ms{nondet}) \tref \letexpr{x}{\ms{nondet}}{(x, x)}
\]
is strict, since the left-hand side can yield arbitrary pairs $(x,y)$, unlike the right-hand side.

Additionally, UB is eliminable (the expression $\letexpr{x}{\ms{div}\;y\;z}{()}$ refines trivially
to $()$), but not introducable—clearly, $() \not\tref \letexpr{x}{\ms{div}\;y\;z}{()}$. Conversely,
nondeterminism is both eliminable and introducable, hence discardable. The nuanced differences
between UB (copyable and eliminable) and nondeterminism (fusable and discardable) capture essential
semantic distinctions between undefined and unspecified behavior.

Commutativity properties further enrich the structure of effects. For example, UB acts as a
right-mover: the refinement $\letexpr{x}{\ms{div}\;y\;z}{(e,x)} \tref (e,\ms{div}\;y\;z)$ holds
universally. However, UB is a left-mover only relative to specific effects, like nondeterminism or
memory access, but fails for effects that alter control flow, such as exceptions or nontermination.
Thus, UB commutes fully with nondeterminism and memory access, but not in general, classifying it as
partially commutative but not central.

In this example, therefore, if $a$ exhibits potential UB, then:
\begin{itemize}
  \item $\letexpr{x}{a}{b} \tref [a/x]b$ for arbitrary $b$
  \item $[a/x]b \tref \letexpr{x}{a}{b}$, and therefore $\letexpr{x}{a}{b} \teqv [a/x]b$, if $x$ is
  used relevantly in $b$ (i.e. at least once) and $b$ has an effect which is a right-mover w.r.t UB
  (for example, nondeterminism, but \emph{not} nontermination).
\end{itemize}

In this talk, we will introduce a formal categorical framework for reasoning about effects in this
manner in the presence of arbitrary control flow. We then introduce \subiterexp{}, a sound and
complete calculus for our categorical semantics. Crucially, our framework supports diverse concrete
models, including sophisticated side-effects such as release-acquire weak memory as formalized in
\citet{release-acquire}.

This work thus offers a unified algebraic foundation for compiler optimization verification,
enabling generalized and compositional reasoning about programs exhibiting complex side effects.

\bibliographystyle{ACM-Reference-Format}
\bibliography{references}

\end{document}