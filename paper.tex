%% Commands for TeXCount
%TC:macro \cite [option:text,text]
%TC:macro \citep [option:text,text]
%TC:macro \citet [option:text,text]
%TC:envir table 0 1
%TC:envir table* 0 1
%TC:envir tabular [ignore] word
%TC:envir displaymath 0 word
%TC:envir math 0 word
%TC:envir comment 0 0

\documentclass[acmsmall,screen,review]{acmart}

\usepackage{syntax}
\renewcommand{\syntleft}{\normalfont\itshape}
\renewcommand{\syntright}{\normalfont\itshape}

\usepackage{prftree}

\usepackage{listings}
\usepackage{xcolor}
\usepackage{subcaption}
\usepackage{fancyvrb}

\definecolor{codegreen}{rgb}{0,0.6,0}
\definecolor{codegray}{rgb}{0.5,0.5,0.5}
\definecolor{codepurple}{rgb}{0.58,0,0.82}
\definecolor{backcolour}{rgb}{0.95,0.95,0.92}

\lstdefinestyle{mystyle}{
%    backgroundcolor=\color{backcolour},   
    commentstyle=\color{codegreen},
    keywordstyle=\color{magenta},
    numberstyle=\tiny\color{codegray},
    stringstyle=\color{codepurple},
    basicstyle=\ttfamily\footnotesize,
    breakatwhitespace=false,         
    breaklines=true,                 
    captionpos=b,                    
    keepspaces=true,                 
    numbers=left,                    
    numbersep=5pt,                  
    showspaces=false,                
    showstringspaces=false,
    showtabs=false,                  
    tabsize=2
}

\lstset{style=mystyle}

\newcounter{todos}
\newcommand{\TODO}[1]{{
  \stepcounter{todos}
  \begin{center}\large{\textcolor{red}{\textbf{TODO \arabic{todos}:} #1}}\end{center}
}}
\newcommand{\sorry}{\textcolor{red}{\textbf{sorry}}}

\newcommand{\todo}[1]{\stepcounter{todos} \textcolor{red}{TODO \arabic{todos}: #1}}

% Math fonts
\newcommand{\mc}[1]{\ensuremath{\mathcal{#1}}}
\newcommand{\mb}[1]{\ensuremath{\mathbf{#1}}}
\newcommand{\ms}[1]{\ensuremath{\mathsf{#1}}}

% Math
\newcommand{\nats}{\mathbb{N}}

% Syntax atoms
\newcommand{\lbl}[1]{{`#1}}
\newcommand{\lto}{\Rightarrow}
\newcommand{\linl}[1]{\ms{inl}\;{#1}}
\newcommand{\linr}[1]{\ms{inr}\;{#1}}
\newcommand{\labort}[1]{\ms{abort}\;{#1}}

% Syntax
\newcommand{\letexpr}[3]{\ensuremath{\ms{let}\;#1 = #2\;\ms{in}\;#3}}
\newcommand{\letstmt}[3]{\ensuremath{\ms{let}\;#1 = #2; #3}}
\newcommand{\brb}[2]{\ms{br}\;#1\;#2}
\newcommand{\lbrb}[2]{\brb{\lbl{#1}}{#2}}
\newcommand{\ite}[3]{\ms{if}\;#1\;\{#2\}\;\ms{else}\;\{#3\}}
\newcommand{\case}[5]{\ms{case}\;#1\;\{\linl{#2} \lto #3, \linr{#4} \lto #5\}}
\newcommand{\where}[2]{#1\;\ms{where}\;\{#2\}}
\newcommand{\wbranch}[3]{#1(#2) \lto #3}
\newcommand{\lwbranch}[3]{\wbranch{\lbl{#1}}{#2}{#3}}
\newcommand{\bsplice}[3]{#1(#2)\;\{#3\}}
%\newcommand{\lbsplice}[3]{\bsplice{\lbl{#1}}{#2}{#3}}
\newcommand{\csplits}[3]{#1 \mapsto #2;#3}
\newcommand{\cwk}[2]{#1 \mapsto #2}
\newcommand{\lwk}[2]{#1 \rightsquigarrow #2}
\newcommand{\tlin}[2]{#2 \subseteq \ms{lin}(#1)}
\newcommand{\ltlin}[3]{#3 \subseteq \ms{lin}(#1) \cap #2}
\newcommand{\thyp}[3]{#1: {#2}^{#3}}
\newcommand{\lhyp}[3]{#1[#2](#3)}
\newcommand{\llhyp}[3]{\lhyp{\lbl{#1}}{#2}{#3}}
\newcommand{\rle}[1]{{\scriptsize\textsf{#1}}}
\newcommand{\taff}{{\{\ms{a}\}}}
\newcommand{\trel}{{\{\ms{r}\}}}
\newcommand{\tint}{{\{\ms{a}, \ms{r}\}}}
\newcommand{\hasty}[5]{#1 \vdash_{#2} #3: {#4}^{#5}}
\newcommand{\haslb}[3]{#1 \vdash #2 \rhd #3}
\newcommand{\lhaslb}[3]{#1 \vdash #2 \rhd #3}
\newcommand{\issubst}[3]{#1: #2 \mapsto #3}
\newcommand{\lbsubst}[3]{#1: #2 \rightsquigarrow #3}
\newcommand{\exprletsubst}[2]{{#1};{#2}}
\newcommand{\stmtletsubst}[2]{{#1};{#2}}
\newcommand{\mhole}[1]{{#1}^?}
\newcommand{\lhole}[1]{?#1}
\newcommand{\mhasty}[6]{#1;#2 \vdash_{#3} #4: {#5}^{#6}}
\newcommand{\mhaslb}[4]{#1;#2 \vdash #3 \rhd #4}
\newcommand{\mlhaslb}[4]{#1;#2 \vdash #3 \rhd #4}
\newcommand{\tyhole}[5]{#1: #2 \mapsto_{#3} {#4}^{#5}}
\newcommand{\blkhole}[3]{#1: #2 \mapsto #3}
\newcommand{\cfghole}[3]{#1: #2 \mapsto #3}
\newcommand{\substctx}[2]{{#1}^{#2}}
\newcommand{\substlbs}[2]{{#1}^{#2}}
\newcommand{\restrictsubst}[2]{{#1}_{#2}}
\newcommand{\subsubst}[2]{#1 \subseteq #2}
\newcommand{\isrw}[3]{#1: #2 \mapsto #3}
\newcommand{\mbind}{\mathbin{{>}\hspace{-0.1em}{>}\hspace{-0.1em}{=}}}
% \newcommand{\strictlbsubst}[3]{#1: #2 \rightsquigarrow_= #3}

% Denotational semantics
\newcommand{\dnt}[1]{\llbracket{#1}\rrbracket}
\newcommand{\ednt}[1]{\left\llbracket{#1}\right\rrbracket}
\newcommand{\upg}[2]{{#1}^{\uparrow #2}}

% Weak memory
\newcommand{\bufloc}[1]{\overline{#1}}

% Branding
\newcommand{\isotopessa}{\ms{isotope_{SSA}}}

%% Rights management information.  This information is sent to you
%% when you complete the rights form.  These commands have SAMPLE
%% values in them; it is your responsibility as an author to replace
%% the commands and values with those provided to you when you
%% complete the rights form.
\setcopyright{acmcopyright}
\copyrightyear{2024}
\acmYear{2024}
\acmDOI{XXXXXXX.XXXXXXX}

%%
%% These commands are for a JOURNAL article.
% \acmJournal{JACM}
% \acmVolume{37}
% \acmNumber{4}
% \acmArticle{111}
% \acmMonth{8}

%%
%% Submission ID.
%% Use this when submitting an article to a sponsored event. You'll
%% receive a unique submission ID from the organizers
%% of the event, and this ID should be used as the parameter to this command.
%%\acmSubmissionID{123-A56-BU3}

\begin{document}

\title{The Denotational Semantics of SSA}

\author{Jad Ghalayini}
\email{jeg74@cl.cam.ac.uk}
\orcid{0000-0002-6905-1303}

\author{Neel Krishnaswami}
\email{nk480@cl.cam.ac.uk}
\orcid{0000-0003-2838-5865}

\begin{abstract}
  ...
\end{abstract}

\begin{CCSXML}
  <ccs2012>
  <concept>
  <concept_id>10003752.10010124.10010131.10010133</concept_id>
  <concept_desc>Theory of computation~Denotational semantics</concept_desc>
  <concept_significance>500</concept_significance>
  </concept>
  <concept>
  <concept_id>10003752.10010124.10010131.10010137</concept_id>
  <concept_desc>Theory of computation~Categorical semantics</concept_desc>
  <concept_significance>500</concept_significance>
  </concept>
  <concept>
  <concept_id>10003752.10003790.10011740</concept_id>
  <concept_desc>Theory of computation~Type theory</concept_desc>
  <concept_significance>500</concept_significance>
  </concept>
  <concept>
  <concept_id>10003752.10003790.10011742</concept_id>
  <concept_desc>Theory of computation~Separation logic</concept_desc>
  <concept_significance>300</concept_significance>
  </concept>
  </ccs2012>
\end{CCSXML}

\ccsdesc[500]{Theory of computation~Denotational semantics}
\ccsdesc[500]{Theory of computation~Categorical semantics}
\ccsdesc[500]{Theory of computation~Type theory}
\ccsdesc[300]{Theory of computation~Separation logic}

%%
%% Keywords. The author(s) should pick words that accurately describe
%% the work being presented. Separate the keywords with commas.
\keywords{SSA, Categorical Semantics, Elgot Structure, Effectful Category}

% \received{20 February 2007}
% \received[revised]{12 March 2009}
% \received[accepted]{5 June 2009}

\maketitle

\cite{sparky}

\bibliographystyle{ACM-Reference-Format}
\bibliography{references}

\clearpage 

\appendix

\begin{figure}
  \begin{center}
    \begin{grammar}
      <\(A, B, C\)> ::= 
      \(X\)
      \;|\; \(A \otimes B\)
      \;|\; \(\mathbf{1}\)
      \;|\; \(A + B\)
      \;|\; \(\mathbf{0}\)

      <\(a, b, c, e\)> ::= \(x\) 
      \;|\; \(f\;a\)
      \;|\; \((a, b)\) 
      \;|\; \(()\) 
      \;|\; \(\linl{a}\) 
      \;|\; \(\linr{a}\)
      \;|\; \(\labort{a}\)
      % \;|\; \(\letexpr{x}{a}{e}\)
      % \;|\; \(\letexpr{(x, y)}{a}{e}\)
      % \;|\; \(\lbsplice{\ell}{x: A}{t}\)
      
      <\(s, t\)> ::= \(\lbrb{\ell}{a}\) 
      \;|\; \(\case{e}{x}{s}{y}{t}\)
      \;|\; \(\letstmt{x}{a}{t}\)
      \;|\; \(\letstmt{(x, y)}{a}{t}\)
      \;|\; \(\where{t}{L}\)

      <\(L\)> ::= \(\cdot\) \;|\; \(\lwbranch{\ell}{x: A}{t}, L\)

      <\(\Gamma\)> ::= \(\cdot\) \;|\; \(\Gamma, \thyp{x}{A}{\epsilon}\)

      <\(\ms{L}\)> ::= \(\cdot\) \;|\; \(\ms{L}, \lbl{\ell}(A)\)
    \end{grammar}
  \end{center}
  \caption{Grammar for \isotopessa, parametrized over a set of instructions \(f \in \mc{I}\)}
  \Description{Grammar for isotope-SSA}
  \label{fig:ssa-grammar}
\end{figure}

\end{document}
\endinput
